\documentclass{aq-notes}
\usepackage{formal}
\usepackage{amsmath, amssymb}
\newcommand{\Range}{\mbox{\normalfont Range}}
\newcommand{\Null}{\mbox{\normalfont Null}}
\newcommand{\Span}{\mbox{\normalfont span}}
\title{\bf Jordan start with nilpotent}
\course{MAT247}
\author{Kunlong Wu}
\begin{document}
\section{Base for nilpotent operators}

\begin{definition}[basic]
    We say $N$ ``basic'' with $(e_1,...,e_2)$ if the basis $(e_1,...,e_n)$ satisfy
    \[Ne_1= 0,Ne_2 = e_1,...,Ne_n= e_{n-1}\]
\end{definition}
\begin{lemma}
    Suppse $\dim \Null(N^i) = i$ for $i\in \{1,...,n\}$, then
    \[N \mbox{ is basic with some basis}\]
    \begin{proof} proof by two steps.
        \begin{itemize}
            \item {\scshape Step 1:} Claim: for $1\leq i\leq n$, $N(\Null(N^i)) = \Null (N^{i-1})$
            \begin{proof}[Proof of step 1]
                If $v\in \Null N^i$, then $0 = N^iv = N^{i-1}(Nv)$. Thus, $Nv\in \Null N^{i-1}$.\\ 
                i.e, $N(\Null(N^i)) \subseteq \Null (N^{i-1})$ \\
                Denote $N_0\in \mathcal{L}(\Null N^i, \Null N^{i-1})$ as $N_0v = Nv$
                \begin{align*}
                    \dim \Range N_0 + \dim \Null N_0 = \dim \Null N^i\\
                    \dim \Range N_0 + 1 = \dim \Null N^i
                \end{align*}
                Therefore $\dim \Range N_0 = i-1$, it follows that $N_0$ is surjective. i.e, $\Null (N^{i-1})\subseteq N(\Null(N^i))$
            \end{proof}
            \item {\scshape Step 2:} Let $e_1\in \Null (N)\setminus \{0\}$.\\
            Pick $e_2 \in \Null (N^2)$ such that $Ne_2 = e_1$. Repeat picking, we get
            \[E  = (e_1,...,e_n) \mbox{ such that } e_i \in \Null(N^i), Ne_i = e_{i-1} i\geq 2\]
            rewrite it, we see there is exists $e$ such that $E= (e,Ne,...,N^{n-1}e)$
            \vspace{1em}
            \item {\scshape Step 3:} Check it is linearly independent.\\
            Suppose it is not linearly independent, then there is a non-trivial solution $a_1,...,a_n$ such that
            \[\sum_{i=1}^na_ie_i = 0\]
            Let $m$ be the largest index such that $a_m \neq 0$.\\
            Then \[0 = \sum a_ie_i \Rightarrow 0 = N^{m-1}\sum a_ie_i = a_me_1\] Contradiction.
        \end{itemize}
    \end{proof}
\end{lemma}
\section{Jordan for Nilpotent}
    \begin{theorem}\label{theo:jordan nilp}
        Suppose $V\in \mathbf{F}^n$ be f.d.v.s., $N\in \mathcal{L}(V)$ is a nilpotent, then we can decompose $V$ into subspace $V_1,...,V_m$ such that 
        \[ V = \bigoplus_{i=1}^m V_i\]
        \begin{enumerate}
            \item $N$ preserves each $V_i$
            \item $N\vert {V_i}$ are basic nilpotent operator
        \end{enumerate}
    \end{theorem}

    \begin{example}
        $\dim V = 3,N^2 =0, \dim \Range(N) = 1$\\
        Suppose \[\Range(N) = \bigoplus_{i=1}^m\Range(N\vert {V_i})\]
        Therefore, wlog $\Range (N\vert {V_1}) = \Range(N)$ and $N\vert {V_i} = 0\vert {V_1}$ for $i > 1$.\\
        Denote $e,f$ as $\Range (N) = \Span (e)$, and $e = Nf$, then $V_1:= \Span(e,f)$\\
        We know that $\Null (N) = 2$ and $e\in \Null(N)$, then denote $g$ as $\Null(N) = \Span(e,g)$. Then $V_2 := \Span(g)$
    \end{example}

    \begin{proof}[Proof of Theorem \ref{theo:jordan nilp}]
        Proceed by induction on $\dim V$.
        \begin{itemize}
            \item {\scshape Step 1}: $\dim \Range(N)< \dim V$, and $N$ \textbf{preserves} $\Range(N)$. \\
            Thus, we apply induction to $(\Range(N), N\vert {\Range(N)})$
            \item {\scshape Step 2}: By induction hypthesis, we can find a basis for $\Range(N)$ as follows:
            \[e_1,...,e_k\mbox{ such that }\Range(N) = \bigoplus_{i=1}^kW_i\ ,\] where $W_i = \Span(e_i,Ne_i,...N^{m_i}e^i)$, where $m_i$ satisfy $N^{m_i}e_i \neq 0$ but $N^{m_i+1}e_i = 0$ \\
            Denote $\epsilon = \{N^je_i | \forall 0\leq j\leq m_i, 1\leq i \leq k)\}$. Then we see that 
            \[\Range(N) = \Span(r) \]
            \item {\scshape Step 3}: Let $\sigma = (f_1,...,f_k), f_i\in V$, be such that $Nf_i = e_i$.\\
            Claim $(\epsilon, \sigma)$ are linearly independent.
            \begin{proof}[Proof of the claim]
                We already know that $\epsilon$ is linearly independent, so suppose $f_1$ is involved, i.e., has non-zero coefficient.\\
                Then, after apply $N$, we find a linearly dependent that does not involving $f$s, which comes a Contradiction.
            \end{proof}
            \item {\scshape Step 4}: For $1\leq i\leq k$, define \[V_i := \Span(f_i,e_i,Ne_i,...,N^{m_i}e_i) = \Span(f_i,Nf_i,...,N^{m_i+1}f_i),\] where $N^{m_i+2}f_i = 0$
            \item Let $f_{k+1},...,f_m\in V$ be such that $(f_1,...,f_m,e_1,...,N^{m_1}e_1,...,e_k,...,N^{m_k}e_k)$ is the basis of $V$.\\
            Note: for each $j$ such that  $k+1\leq j\leq m$, we have
            \[Nf_j \in \Range(N) = \Span(N^je_i)\]
            Therefore, there exists $g_j\in \Range(N)$ such that $Ng_j = Nf_j$. Define $f_j' := f_j-g_j$, then 
            \[(f_1,...,f_k,f'_{k+1},...,f'_m,e_1,...,N^{m_1}e_1,...,e_k,...,N^{m_k}e_k)\mbox{ is still a basis of }V\]
            \item {\scshape Step 5}: For $k+1\leq j \leq m$, define $V_j := \Span(f'_j)$, then
            \[V = \bigoplus_{i=1}^m V_j\]
            \[N\vert {V_j} \mbox{ is a basis for each }j\]
        \end{itemize}
        
    \end{proof}
\section{Extend Jordan form to all linear operators}
    Let $T\in \mathcal{L}(V)$ be any arbitrary linear operator over $V$.
    \[V = \bigoplus_{j=1}^kG(\lambda_j,T)\]
    Then, we see that 
    \[(T-\lambda_jI)\vert {G(\lambda_j,T)} \mbox{ is nilpotent}\]
    thus they can put it in Jordan form and preserves in there generalized eigenspace.\\
    Therefore, $T$ can generate a Jordan form matrix.

\section{Invariant of Jordan Form}
\begin{theorem}
    Let $V$ be f.d.v.s., $N\in \mathcal{L}(V)$ be nilpotent operator
    \[\exists V = \bigoplus_{i=1}^mV_i. \mbox{ such that $N$ preserves $V_i$.}\]
    Let $a_j$ be the number of terms in the decomposition of dimention $j$.\\
    The sequence $(a_1,...,a_j)$ is determined by $N$.
\begin{proof}

Suppose $V = \Span(e_1,...,e_k)$, such that $Ne_1=0, Ne_i = e_{i-1}$. Then we see that
\[\dim \Null(N) = 1, \dim \Null(N^2) = 2,\ \cdots,\dim \Null(N^k) = k, \dim \Null(N^{k+1}) = k.\]
Therefore, if $(W,N)$ is such that $\dim W = k$ and N is a basic nilpotent, then \[\dim \Null(N^j) = \min(j,k).\]
\begin{align*}
    \dim \Null N^j &= \sum\dim \Null N^j\vert {V_i}\\
&=a_1\min(j,1)+a_2\min(j,2)+\cdots
\end{align*}
Define $n_j  = \dim\Null N^j$, then
\begin{align*}
    n_1 &= a_1+a_2+\cdots\\
    n_2 &= a_1+2a_2+2a_3+\cdots\\
    n_3 &= a_1+2a_2+3a_3+3a_4+\cdots
\end{align*}
Then, subtract consecutive equations
\begin{align*}
    n_2-n_1 &= a_2+a_3+\cdots\\
    n_3-n_2 &= a_3+a_4+\cdots\\
    n_4-n_3 &= a_4 + a_5+ \cdots
\end{align*}
Then we got
\begin{align*}
    	2n_1-n_2&= a_1\\
    	-n_1+2n_2-n_3 &= a_2\\
    	\vdots\\
    	-n_k+2n_{k+1}-n_{k+2} &= a_{k+1}
\end{align*}
\end{proof}
\end{theorem}


\begin{corollary}\label{cor:observation}
    \[\dim\Null(N^{{j+2}}) - \dim\Null (N^{j+1})\leq \dim \Null N^{j+1}-\dim\Null N^j\]
\end{corollary}
\begin{example}
    If $\dim\Null N = 3,\dim \Null N^2 = 5$, then $\dim\Null N^4 \leq 9$
\end{example}


\begin{proof}[Alternative proof of \ref{cor:observation}]
    Idea: Apply rank-Nullity thm to subspaces of V.\\    
    It is easy to see that $\Null(N^j\vert {\Null(N^{j+1})}) =\Null N^j$\\
    Thus \[\dim\Null N^{j+1} - \dim\Null N^j = \dim \Range(N^j\vert {\Null(N^{j+1})})\]
    \[\dim\Null N^{j+2} - \dim\Null N^{j+1} = \dim \Range(N^{j+1}\vert {\Null(N^{j+2})})\]
    Claim: $\Range(N^{j+1}\vert {\Null(N^{j+2})})\subset \Range(N^j\vert {\Null(N^{j+1})})$
\begin{proof}[proof of claim]
        Let $v\in \Null(N^{j+2}) \Rightarrow Nv\in \Null(N^{j+1})$.Then, 
        \[N^{j+1}v = N^j(Nv)\in \Range(N^j\vert {\Null(N^{j+1})})\]
\end{proof}Therefore,
\[\dim \Range(N^{j+1}\vert {\Null(N^{j+2})})\leq \dim\Range(N^j\vert {\Null(N^{j+1})})\]
\end{proof}

\end{document}